\documentclass{article}

\usepackage[utf8]{inputenc}
\usepackage[T1,T2A]{fontenc}
\usepackage[english,russian]{babel}
\usepackage{expdlist}

\usepackage{amsfonts}
\usepackage{amsmath}
\usepackage{amssymb}
%\usepackage{amsthm}
\usepackage{ifpdf}
\usepackage{moreverb}
\usepackage{enumerate}
\def\verbatimtabsize{4\relax}

\emergencystretch=25pt

\usepackage{listings}
\lstloadlanguages{C,[ANSI]C++,Clean,make,Fortran}%Загружаемые языки
\lstset{extendedchars=true, %Чтобы русские буквы в комментариях были
        commentstyle=\it,
        stringstyle=,
        language=C++, %Язык по умолчанию
        belowcaptionskip=5pt}

\ifpdf
    \usepackage[pdftex]{graphicx}
\else
    \usepackage{graphicx}
\fi

\usepackage{geometry}
\geometry{left=2cm}
\geometry{right=1.5cm}
\geometry{top=1cm}
\geometry{bottom=2cm}

\setlength{\parindent}{1cm}

\newcommand{\set}[1]{\left \{ #1 \right \}}
\newcommand{\setst}[2]{\left \{ #1 \mid #2 \right \}}
\newcommand{\abs}[1]{\left| #1 \right|}

\providecommand{\R}{\mathbb{R}}
\providecommand{\Q}{\mathbb{Q}}
\providecommand{\Z}{\mathbb{Z}}
\providecommand{\markthis}[1]{{\tt[#1]}\par}

\newtheorem{theorem}{Теорема}
\newtheorem{exttheorem}{Теорема}
\newtheorem{lemma}[theorem]{Лемма}
\newtheorem{corollary}[theorem]{Следствие}
\newtheorem{definition}[theorem]{Определение}
\newtheorem{claim}[theorem]{Утверждение}

\newcommand{\refeq}[1]{(\ref{eq:#1})}
\newcommand{\refth}[1]{теореме~\ref{th:#1}}
\newcommand{\reflm}[1]{лемме~\ref{lm:#1}}
\newcommand{\refsec}[1]{разделе~\ref{sec:#1}}
\newcommand{\refcor}[1]{следствии~\ref{cor:#1}}
\newcommand{\refalgo}[1]{алгоритме~\ref{algo:#1}}

\newenvironment{proof}{\par\noindent%
{\bf Доказательство.\par\nopagebreak}}{\unskip\nobreak\enskip$\square$\par\bigskip}
\newenvironment{proofof}[1]{\medskip\par\noindent%
{\bf Доказательство
#1.\par\nopagebreak}}{\unskip\nobreak\enskip$\square$\par\bigskip}

\newcommand{\rk}{\mathop{\mathrm{rk}}\nolimits}
\newcommand{\sort}{\mathsf{sort}}
\newcommand{\evens}{\mathsf{evens}}
\newcommand{\med}{\mathsf{med}}

\renewcommand{\epsilon}{\varepsilon}

\frenchspacing
\sloppy

\begin{document}



\subsection{Модель вычислений}
% What about vanila stream model?
В данной работе мы будем работать с большими объемами данных.
Причем данные будут представлены некоторым потоком, и,
в отличии от алгоритмов во внешней памяти, мы будем рассматривать
алгоритмы, которые не пользуются внешней памятью.
Более конкретно, все наши алгоритмы будут устроены следующим образом.
На вход поступает некоторый поток чисел, который читается один раз,
наша задача, используя минимальное количество памяти и времени,
посчитать некоторую функцию $f$.  Причем, как правило, функция
зависит лишь от множества чисел, которые содержаться в потоке и не зависит от их порядка.

Обозначим, поток числе за $\sigma = \langle a_1, \ldots, a_m \rangle$, где для всех $i$
верно, что $a_i \in [1, \ldots, n]$.

Как правило числа $n$ и $m$ очень большие (например, $n = 2^{64}$ и $m = 10^{12}$).
Наша цель придумать алгоритм, время работы которого состовляет $o\left(\min(n, m)\right)$.
Наилучшая оценка, которую хочется достичь,~--- это $O(\log n + \log m)$, однако это,
как правило невозможно.

Для многих задач можно доказать, что нижняя оценка на время работы состовляет $\Omega(n + m)$.
Поэтому часто удобно ослабить модель. Это можно сделать двумя способами:
\begin{enumerate}
\item
Разрешить делать более одного прохода по потоку.

\item
% Улучшить обозначения
Решать задачу приближенно, то есть вычислить функцию $\hat{f}(\sigma)$,
зависящую от параметров $\epsilon$ и $\delta$, которая удовлетворяет условию
$P \left( \abs{f - \hat{f}}  > f \cdot \epsilon \right) < \delta$.
\end{enumerate}



\subsection{Примеры алгоритмов}

Первый момент, сумма чисел. Необходимо просто сложить все числа.
Очевидно, что задача решается с использованием памяти $O(\log m + \log n)$.

Нулевой момент, количество различных чисел в потоке. Это вам расскажет/рассказал Максим.

Нижние оценки.
\begin{theorem}
    Любой детерминированный алгоритм, решающий задачу поиска количества различных элементов
    требует $\Omega(n)$ памяти.
\end{theorem}

\begin{proof}
    % Изложить доказательство более понятным и последовательным языком.
    % Дать ссылку на задачу про EQ
    Воспользуемся тем, что коммуникационная сложность задача \textsc{EQ} состовляет $n$.
    Пусть мы научились решать задачу поиска различных элементов за время $o(n)$.
    Рассмотрим экземпляр задачи \textsc{EQ}, то есть две бинарные строки $x$ и $y$ длины $n$.
    Составим из них поток чисел $\sigma$ длины $2n$:
    $$
        a_i = 
        \begin{cases}
            x_i + 2 * (i - 1) & \text{если $1 \leq i \leq n$} \\
            y_i + 2 * (i - n - 1) & \text{если $n + 1 \leq i \leq 2n$}
        \end{cases}
    $$
    Заметим, что $x = y$, тогда и только тогда, когда в потоке $n$ различных чисел.
    Покажем, как, используя решение задачи \textsc{количество различных}, получить решение задачи
    \textsc{EQ}. Пусть и Алиса и Боб знают алгоритм $A$. Алиса преобразует свое $x$ в
    соотвествующую часть потока чисел, после чего посылает Бобу конечное состояние алгоритма.
    Боб продолжает исполнение алгоритма на своей послоедовательности и в конце получает ответ про количество
    различных элементов, то есть про равенство $x$ и $y$, противоречие.
\end{proof}

\begin{theorem}
    % разобраться с доказательством этого факта
    Задача \textsc{большинство} требует $\Omega(m)$ памяти.
\end{theorem}



\subsection{Поиск медианы}

Рассмотрим следующую задачу, как обычно дан поток чисел, требуется
вычислить медиану. Известно, что в обычном случае эту задачу можно решить
за линейное время с помощью алгоритма Таржана деления на пятерки.
Однако этот алгоритм требует использование дополнительного места.
% Продумать историю насчет того, что впринципе данный алгоритм применим,
% однако требует дополнительно двух хранилищ памяти. И вообще говоря хочется
% обойтись без них (представьте, что у вас тонкий клиент, и вы делаете запрос
% в огромную базу данных. Однако, в таком случае более оптимален рандомизиоронный алгоритм

\begin{theorem}
    Поиск медианы в потоке за один проход требует $\Omega(n)$ памяти.
\end{theorem}

\subsubsection{Алгоритм Манро-Патерсона}

Но задачу можно решить эффективно, если разрешается читать поток несколько раз.
За два прохода задачу можно решить за время $\tilde{O}\left(\sqrt n\right)$.
За $p$ проходов задачу можно решить за время $\tilde{O}\left(n^{1 / p}\right)$.
То есть за $\log n$ проходов можно добиться использования логарифмической памяти.

\begin{definition}
    Рассмотрим поток чисел $P$ размера $s \cdot 2^i$, где число $s$ четное.

    $0$-выборка потока $P$~--- это $\sort(P)$.

    $i$-выборка потока $P = A \circ B$~--- это $\sort \left(\evens(A) \cup \evens(B) \right)$
\end{definition}

По определению $i$-выборка потока имеет размер $s$. Заметим, что $i$-выборку можно
вычислить используя $O\left(s \cdot \log |P| \right)$ памяти. Достаточно на каждом уровне
$0 \leq k \leq i$ поддерживать не более одной выборки.
Как только становится две выборки будем объединять их в одну $(k+1)$-выборку.
То есть если у нас имеется $S$ памяти, возьмем $s$ равным $S / \log |P|$.

\begin{claim}
    Рассмотрим $p$-ый элемент $i$-выборки тогда для него верны следующие оценки на 
    позицию в $\sort(P)$. Оценка снизу на элемент равна $L_{pi} = p 2^i$, оценка
    сверху равно $U_{pi} = (p + i) 2^i$.
\end{claim}

\begin{proof}
\end{proof}

Опишем алгоритм. На каждом шаге будем поддерживать два числа $a$ и $b$, нижнюю
и верхнюю границы на медиану, соответственно, то есть $a \leq \med(P) \leq b$.
Выполним проход по потоку и вычислим новые границы на медиану $a' > a$ и $b' < b$.

Вычислять границы будем следующим образом. Вычислим $i$-выборку подпотока, лежащую
между элементами $a$ и $b$. Обозначим $sm = [s / 2]$, рассмотрим элементы
$x_{(sm-i) i}$ и $x_{sm i}$, согласно утверждению эти элементы дают новую границу
на медиану. Посмотрим на нижнюю и верхнюю границу новых элементов, и получим, что
$$
m' = b' - a' \leq 2 i 2^i = m \cdot \frac{2i}{s} = m \cdot \frac{2i \log m}{S} 
    = O\left(m \frac{\log^2 m}{S} \right)
$$

Здесь мы воспользовались тем, что $i$ не превосходит $\log m$.
Также, мы переобозначили $m$ за $|P|$. Таким образом, если $S > \sqrt{m} \log^2 m$,
то мы получим алгоритм работающий за два прохода.

Заметим, что данный алгоритм работает не только для поиска медианы, но и для
поиска любой порядковой статистики. Для этого надо соотвествующим образом выбирать
элементы $i$-выборки для пересчета новых границ $a'$ и $b'$.


\subsubsection{Алгоритм \textsc{count-min}}
Будем решать задачу поиска медианы в один проход. Мы знаем, что решить
точно задачу не получится, поэтому будет решать её приближенно.

Для этого решим более общую задачу. Пусть имеется массив чисел $a$
размера $n$, заполненный единиц. Также, пусть есть поток запросов $(i, c_i)$:
прибавить $i$-ому элементу массива $c_i$.
Причем будет работать в Cash Registry случае, то есть в любой момент
элементы массива не отрицательны.

Хочется научиться отвечать на следующие запросы:
\begin{itemize}
\item
$Q(i)$~--- выдать $i$-ый элемент массива.

\item
$Q(l, r)$~--- выдать сумму элементов массива с $l$-го по $r$-ый.

\item
$a \circ b$~--- вычислить поэлементное произведение двух массивов $a$ и $b$.
\end{itemize}

Хочется, чтобы структура данных имела размер сильно меньше $n$.
Поэтому будем решать задачу приближенно, выберем параметры $\epsilon$ и $\delta$.
Заведем таблицу $c$ размеров 
$d = \lceil \log \frac{1}{\delta} \rceil$ на
$w = \lceil \frac{e}{\epsilon} \rceil$, заполненную нулями.
Выберем $d$ 2-независимые хэш-функции $h_1, \ldots, h_d$.
Когда приходит запрос выполним следуюшие действия:
$$
c[j][h_j[i]] += c_i
$$




\end{document}

