\documentclass{article}

\usepackage[utf8]{inputenc}
\usepackage[T1,T2A]{fontenc}
\usepackage[english,russian]{babel}
\usepackage{expdlist}

\usepackage{amsthm}
\usepackage{amsfonts}
\usepackage{amsmath}
\usepackage{amssymb}
\usepackage{ifpdf}
\usepackage{moreverb}
\usepackage{enumerate}
\usepackage{comment}
\def\verbatimtabsize{4\relax}

\emergencystretch=25pt

\usepackage{listings}
\lstloadlanguages{C,[ANSI]C++,Clean,make,Fortran}%Загружаемые языки
\lstset{extendedchars=true, %Чтобы русские буквы в комментариях были
        commentstyle=\it,
        stringstyle=,
        language=C++, %Язык по умолчанию
        belowcaptionskip=5pt}

\ifpdf
    \usepackage[pdftex]{graphicx}
\else
    \usepackage{graphicx}
\fi

\renewenvironment{proof}{\par\noindent%
{\bf Доказательство.\par\nopagebreak}}{\unskip\nobreak\enskip$\square$\par\bigskip}
\newenvironment{proofof}[1]{\medskip\par\noindent%
{\bf Доказательство #1.\par\nopagebreak}}{\unskip\nobreak\enskip$\square$\par\bigskip}

\newtheorem{mytheorem}{Теорема}
\newtheorem{mylemma}{Лемма}
\newtheorem{myclaim}{Утверждение}
\newtheorem{exersize}{Упражнение}


\begin{document}

\section{Топология}

Напомним основные определения:

Множество $X$ и семейство множеств $\tau$ называется \emph{топологическим пространством},
если выполнены следующие свойства:
\begin{enumerate}
\item
$\varnothing$ и $X$ принадлежат семейству $\tau$.

\item
Семейство $\tau$ замкнуто относительно произвольных объединений.

\item
Семейство $\tau$ замкнуто относительно конечных пересечений.
\end{enumerate}

Множества из семейства $\tau$ называют \emph{открытыми}.

Примеры:
\begin{itemize}
\item
Любое множество, в котором $\tau$~--- это все подмножества.
Например, множество $\mathbb{B}$ или $\mathbb{N}$.

\item
% Открытое множество -- это произвольное объединение цилинров!
\emph{Множество Кантора}, то есть множество бесконечных последовательностей из нулей и единиц.
В качестве $\tau$ возьмем всевозмоные объединения цилиндров. \emph{Цилиндр}~--- это множество
последовательностей с общим префиксом, например, множество всех последовательностей,
начинающихся с $01$.

\item
Множество действительных чисел $\mathbb{R}$, где $\tau$~--- это открытые множества,
в классическом смысле (то есть у каждой точки множества, есть окрестность, которая
полностью принадлежит множеству).
\end{itemize}

\medskip

Множество $S$ из топологического пространства $X$ называется
\emph{компактным}, если из любого покрытия $X$ открытыми множествами можно
выделить конечное подпокрытие.

Часто интересуются вопросом, является ли само топологическое пространство компактным.

Примеры:
\begin{itemize}
\item Конечное множество с полной топологией является компактным.
\item Множество $\mathbb{N}$ не является компактным, например,
    возьмем покрытие одноэлементными множествами.
\item Множество $[0, 1]$ является компактным (см. лекции по матанализу).
\item Канторово множество является компактным, давайте докажем это.
\end{itemize}

% Изложить эти утверждения и доказательства понятней
\begin{mylemma}[Кёнига]
У бесконечного бинарного дерево есть бесконечная ветвь.
\end{mylemma}

\begin{proof}
Предъявим алгоритм построения бесконечной ветви. Начнем построение с корня дерева.
Так как дерево бесконечно, значит либо правое, либо левое поддерево бесконечно.
Пойдем в бесконечное поддерево и продолжим делать тоже самое.
\end{proof}

\begin{myclaim}
Множество Кантора является компактным.
\end{myclaim}

\begin{proof}
Предположим противное, пусть множество Кантора не является компактным.
Рассмотри бесконечное покрытие цилиндрами, которые не допускает выделение конечного подпокрытия.
Рассмотрим бесконечное дерево, отвечающее всему множеству Кантора, и обрежем
его в вершинах, которые представляют цилиндры из покрытия. Так как нет конечного подпкрытия
полученное дерево будет также бесконечным. Применим лемму Кёнига и выделим бесконечную ветвь.
Бесконечная ветвь будет некоторым элементом множества Кантора и не будет покрыта, так как
мы её не обрезали, противоречие.
\end{proof}

\begin{exersize}
Докажите, что множество $\mathbb{N} \to \mathbb{N}$ не является компактным.
\end{exersize}

\medskip

Рассмотрим произвольную функцию $f : X \to Y$, где $X$ и $Y$~--- это топологические пространства.
Функция $f$ называется \emph{непрерывной}, если для любого открытого множества $A$ в пространстве $Y$ 
его прообраз $f^{-1}(A)$ является открытым в пространстве $X$.

Примеры:
\begin{itemize}
\item
Тождественная функция $id: X \to X$.

\item
Любая функция $f : \mathbb{N} \to \mathbb{B}$, так как все подмножества $\mathbb{N}$
являются открытыми.

\item
Функция $f : [0, 1] \to \mathbb{B}$, равная:
$$
f = 
\begin{cases}
0, & \text{если $x \leq 0.5$}; \\
1, & \text{иначе}.
\end{cases}
$$
не является непрерывной, так как прообраз $0$ не является открытым множеством.
\end{itemize}


\subsection{Вычислимые функции}

Выясним, как связана топология с вычислыми функциями.  
Для начала напомним классические определения.

Функция $f: N \to N$ называется \emph{вычислимой}, если существует алгоритм $M$
(например, написанный на машине Тьюринга), который вычисляет функцию $f$
в следующем смысле:
\begin{enumerate}
\item
Если $f(x)$ определено, то за конечное время вычисляется $M(x)$, которое должно быть
равно $f(x)$.

\item
Если $f$ на $x$ не определено, то машина $M$ на входе $x$ не останавливается.

\end{enumerate}

% Почитать про систему типов и понять, что к чему
Заметим, что все машины Тьюринга можно закодировать натуральными числами.
Учитывая этот факт, можно говорить о вычислиых функциях следующего типа:
$f : (\mathbb{N} \to \mathbb{B}) \to \mathbb{B}$. То есть это функции, которые
на вход принимают закодированные вычислимые функции типа $\mathbb{N} \to \mathbb{B}$, 
а на выходе выдают $0$ или $1$. Важно отметить, что в качестве аргумента в данном
случае рассматриваются только вычислимые функции.

% Понять как вычислимость соотносится с непрерывностью
%\begin{myclaim}
%Пространство вычислимых функций 
%\end{myclaim}


\subsection{Проверка функций на равенство}
Рассмотрим вычислимые две произвольные вычислимые всюду определенные функции $f$ и $g$,
имеющие следующий тип $(\mathbb{N} \to \mathbb{B}) \to \mathbb{B}$, то есть функции,
которые сопоставляют 0,1-последовательностям значение 0 или 1.

Оказывается, что задача проверки двух таких функций на равенство вычислима.

\begin{myclaim}
Всюду определенная вычислимая функция $(\mathbb{N} \to \mathbb{B}) \to \mathbb{B}$
читает не более константного количества бит входной последовательности.
\end{myclaim}

\begin{proof}
Воспользуемся тем, что любая вычислимая функция является непрерывной.
Значит прообразы нуля и единицы открыты. Возьмем произвольную последовательность
из нулей и единиц. Она является частью открытого множества, следовательно покрывается
некоторым цилиндром. Следовательно для некоторого $n$ вычисление вычисление
не зависит от битов последовательности больше $n$-го. Обрубим дерево вычисления на $n$-ом
бите и запишем там значение. Проделаем такие обрезания для всевозможных последовательностей.

Утверждается, что получится конечное дерево, вычисляющее нашу функцию.
Пусть дерево получилось бесконечным, тогда по лемме Кенига у него есть бесконечная ветвь,
применим наше обрезание к бесконечной ветве, получим противоречие, следовательно
дерево вычисления функции конечно.

Из конечности дерева следует, что функция читает не более конечного количества
бит входной последовательности.
\end{proof}

Таким образом, чтобы проверить две функции на равенство достаточно лишь проверить, что
конечные деревья вычислений этих функций совпадают. Данный алгоритм можно имплементировать
на любом языке программирования, однако, понадобятся некоторые трюки. Поэтому
мы приведем пример имплементации на языке Haskell.



%Урпажнение:
%
%X is discrete <=> f : X \times X -> B, f(x, x) = 1 | 0 otherwise is continious,

\end{document}

