\documentclass{article}
% Русский
\usepackage[utf8]{inputenc}
\usepackage[T1,T2A]{fontenc}
\usepackage[english,russian]{babel}
\usepackage{expdlist}


% AMS
\usepackage{amsfonts}
\usepackage{amsmath}
\usepackage{amssymb}
\usepackage{amsthm}
\usepackage{ifpdf}
\usepackage{moreverb}
\usepackage{enumerate}
\def\verbatimtabsize{4\relax}

\emergencystretch=25pt

\usepackage{listings}
\lstloadlanguages{C,[ANSI]C++,Clean,make,Fortran}%Загружаемые языки
\lstset{extendedchars=true, %Чтобы русские буквы в комментариях были
        commentstyle=\it,
        stringstyle=,
        language=C++, %Язык по умолчанию
        belowcaptionskip=5pt}

\ifpdf
    \usepackage[pdftex]{graphicx}
\else
    \usepackage{graphicx}
\fi

\newtheorem{claim}{Утверждение}
\renewenvironment{proof}{\par\noindent%
{\bf Доказательство.\par\nopagebreak}}{\unskip\nobreak\enskip$\square$\par\bigskip}
\newenvironment{proofof}[1]{\medskip\par\noindent%
{\bf Доказательство #1.\par\nopagebreak}}{\unskip\nobreak\enskip$\square$\par\bigskip}

\begin{document}

\subsection*{Задача 1.}
\begin{itemize}
    \item
    Покажем как используя одну операцию поиска транзитивного замыкания
    графа перемножить две булевых матрицы $A$ и $B$. Пусть размеры исходных
    матриц $n \times n$, построим граф из $3n$ вершин следующим образом:
    в графе будет три слоя размера $n$. Между первым и вторым слоем
    проведем ребра согласно матрице $A$, между вторым и третьим согласно
    матрице $B$.

    Теперь, если в таком графе найти транзитивное замыкание и составить
    матрицу из ребер между первым и третьим слоем --- получится матрица
    $C = A \cdot B$.

    \item
    Рассмотрим имеющийся граф и его матрицу смежности $A$. Сначала приделаем
    к графу всевозможные петли, после выполнения данной операции
    транзитивное замыкания графа не изменится. Теперь если рассмотреть матрицу
    $A^2$, то её граф будет представлять собой граф матрицы $A$ плюс ребра,
    представляющие пети длины два в начальном графе. Таким образом,
    граф матрицы смежности $A^n$ будет соответствовать транзитивному
    замыканию начального графа. То есть за логарифимическое количество
    перемножений матрицы смежности можно вычислить транзитивное
    замыкание графа.
\end{itemize}

\subsection*{Задача 2.}
\begin{enumerate}[a)]
    \item
    За каждый шаг количество вершин в графе уменьшается не менее, чем вдвое.
    Компоненты связности в графе можно искать за время $O(E)$. Поэтому
    суммарное время работы алгоритма Борувке будет $O(E logV)$.
    \item
    Естественно, что кратные ребра в графе надо удалять. В таком случае
    предыдущая оценка алгоритма не оптимальна, так как после нескольких
    операция стягивания количество ребер в графе будет ограничено не только
    величиной $|E|$, но и величиной $|V_i|^2$. То есть сложность алгоритма
    составляет:
    $$
        \min(V_0^2, E) + \min(V_1^2, E) + \ldots
    $$
    Здесь $V_i = |V| / 2^i$. Пусть первые $k$ шагов минимум
    достигается на значение $E$, а все оставшиеся шаги на значении
    $V_i^2$. Тогда сложность алгоритма составляет:
    $$
        k \cdot E + c \frac{|V|^2}{2^{2k}}.
    $$
    Здесь мы воспользовались тем, что ряд из квадратов степеней двойки
    сходится. Несложными вычислениями можно показать, что максимум
    функции сложности, относительно переменной $k$ достигается при:
    $$
        k = \log \frac{V^2}{E} + O(1)
    $$
    То есть сложность алгоритма Борувке составляет $O(E \log V^2 / E)$.
\end{enumerate}

\subsection*{Задача 3.}
Достаточно провести $\log \log V$ шагов алгоритма Борувке, после чего к
полученному графу применить алгоритм Прима.

\subsection*{Задача 4.}
Рассматрим правильную 3-раскраску графа, назовем её $K$.
Будет следить за количеством $d$ -- отличающихся по цвету вершин
текущей раскраски и правильной. Рассмотрим шаг нашего алгоритма.
Имеется треугольник $(v_1, v_2, v_3)$, у которого случайным образом
мы будем перекрашивать вершину (в этом треугольнике цвет одной вершины
совпадает с $K$, а двух других нет). С вероятностью $1/3$ будет перекрашена
вершина, которая имеет цвет совпадающий с цветов в $K$, и расстояние
до оптимального решения увеличится на единицу. С вероятностью $1/3$
мы перекрасим не совпадающую по цвету вершину в цвет этой вершины в $K$
и расстояние до $K$ уменьшится на единицу. И еще с вероятностью $1/3$
расстояние до $K$ не изменится. Таким образом наш алгоритм соответствует
случайному блужданию по графу являющемуся цепью с петлями, причем вероятности
переходов по всем ребрам одинаковые. Время покрытия такого графа
не превосходит $|V|^2$, таким образом матожидание времени работы
указанного алгоритма не превосходит $|V|^2$ (при этом алгоритм может
остановится раньше, не достигнув правильной 3-раскраски, если в графе
не станет одноцветных теругольников).

\subsection*{Задача 5.}
Рассмотрим множество всех распределений $x$. Оно образует
симплекс в $\mathrm{R}^n$. Матрица $A$ является отображением
и этого в симплекса в его компактное подмножество. По теореме
Брауэра у этого отображение есть неподвижная точка.

\subsection*{Задача 6.}
Эта задача была решена на лекции. Достаточно проверить, что указанное
распределение стационарно и применить основную теорему марковских цепей.

\subsection*{Задача 7.}
Зафиксируем ребро, у которого будет изменяться сопротивление.
Рассмотрим систему уравнений на потенциалы. Приведем её к диагональному
виду так, чтобы последние две строки отвечали концам рассматриваемого
ребра. Тогда изменение проводимости на $\varepsilon$ приведет к тому, что
у элементы системы изменяться следующим образом:
\begin{align}
$\hat{a}_{n-1,n-1} = a_{n-1,n-1} + \varepsilon$ \\
$\hat{a}_{n,n-1} = a_{n,n-1} - \varepsilon$ \\
$\hat{a}_{n-1,n} = a_{n-1,n} - \varepsilon$ \\
$\hat{a}_{n,n} = a_{n,n} + \varepsilon$ \\
\end{align}

Решаю новую систему несложно заметить, что относительно $\varepsilon$
все потенциалы и их разности будут дробно-линейным функциями.
А дробно-линейные функции монотонны на своих учатсках определенности.


\subsection*{Задача 8.}
Эта задача была на лекции, для времени покрытия ответ $n H_n$.

\subsection*{Задача 9.}
Начнем со вспомогательных фактов.

Утверждение: задачу измерения эквивалетнтного сопротивления между
двумя вершинами в графе с единичными сопротивления можно рассматривать
как потоковую задачу со следующими ограничениями. Мы хотим протолкнуть
единицу потока из вершины $s$ в вершину $t$ так, чтобы было верно
следующее условие: для любого $s-t$ пути $P = (e_1, e_2, ..., e_k)$
верно, что $\sum_i f(e_i) = c$, где $c$ -- некоторая константа не зависящая
от выбора пути $P$. Для доказательства данного факта достаточно
положить $f(uv) = \varphi_v - \varphi_u$, то есть в качестве потока
будет выступать перепад потенциалов.

Описанный выше поток существует и единственен всвязном графе.
Существование следует из того,
что система уравнений на потенциалы совместна и вырождена. Докажем
единственность: пусть существует два различным потока, тогда можно
рассмотреть их разность. Разность будет не тождественно нулевой
(то есть она представляет собой некоторую циркуляцию), при этом в графе
нет источников напряжения. Рассмотрим вершину с наименьшим потенциалом.
Так как из вершины не должно ничего вытекать -- отсюда следует, что
у всех её соседей потенциал такой же. Далее по индукции можно показать,
что у всех вершин графа потенциал одинаковый, то есть циркуляция вырождена,
что противоречит предположению.

Величина потока по каждому ребру не превосходит единицы. Декомпозируем
поток, в силу предыдущего утверждения он будет состоять только из путей.
Отсюда все следует.

Эквивалетное сопротивление цепи между не превосходит минимального расстояния
между рассматриваемыми вершинами.

Доказанные выше утверждения помогают решить 9, 10 и возможно 7-ю задачу.

В данной задаче достаточно показать, что для любой пары вершин минимальное
расстояние между ними не превосходит $c n/d$. Тогда время двойного обхода
между этими вершинами не превосходит $n \cdot d \cdot (c n/d) = c n^2$,
откуда по теореме о связи время обхода графа и сопротивления графа
следует, что время обхода графа не превосходит $O(n^2 \log n)$.
Упорядочим вершины по слоям согласно минимальному расстоянию до вершины $s$.
Тогда количество вершин в трех подряд идущим слоях не менее $d$, так как у
вершины из среднего слоя все соседи лежат в данных трех. Значит количество
слоев не превосходи $3 n / d$ и утверждение доказано.



\subsection*{Задача 10.}
Если ребро $(u,v)$ является мостом, то несложно проверить, что
эквивалентное сопротивление между вершинами $u$ и $v$ будет равно
одному. Пользуясь леммой из лекций, отсюда следует, что $C_{uv} = 2 |E|$.

Пусть $(u, v)$ не является мостом, тогда
эквивалентное сопротивление между вершинами $u$ и $v$ будет равно
$$
    \frac{1}{1 + \frac{1}{\hat{R}_{uv}}} < 1
$$
Здесь $\hat{R}_{uv}$ -- это эквивалентное сопротивление между вершинами
$u$ и $v$ в графе без ребра $(u, v)$. Аналогично, пользуясь леммой о связи
эквивалентного сопротивления и $C_{uv}$ получаем, что $C_{uv}$ меньше $2|E|$.

\subsection*{Задача 14.}
Рассмотрим распределение работ при использование жадного алгоритма.
Пусть имеется $n$ работ и $m$ машин, также пусть все работы отсортированы
по невозрастанию длительности $t_1 \geq t_2 \geq \ldots \geq t_n$.
Рассмторим машину, на которой достигается максимум времени работы жадного
алгоритма и рассмотрим последнюю работу на данной машине. Назовем эту
работу $t_k$. Тогда $GREEDY \leq OPT + t_k$, так как в момент когда
добавляли работу $t_k$ все машины были заняты на как минимум $GREEDY - t_k$
единиц времени.
\begin{enumerate}
    \item
    Если $k \leq n$, то $OPT$ равен $GREEDY$.
    Так как в данном случае $OPT \leq GREEDY \leq t_k \leq OPT$.

    \item
    Если $k > 2n$. Рассмотрим распределение работ $t_1, \ldots t_k$
    в оптимальном расписании, хотя бы три из них выполняются на одной машине,
    значит $t_k \leq OPT/3$, и утверждение задачи доказано.

    \item
    Если $n < k \leq 2n$. Пусть $t = GREEDY - t_k$ и $GREEDY > 4/3 OPT$.
    У нас имеется не менее $2 n - k$ машин у которых на момент распределения
    работы $t_k$ имелась всего одна работа. Посмотрим на оптимальное
    распределение работ $t_1 \ldots t_k$, те работы, которые были
    распределены на одну машину в жадном алгоритме также должны будут
    быть распределены и в оптимальном. При этом в оптимальном
    распределении не может быть трех работ на одной машине, так как
    иначе $t_k \leq OPT / 3$ и работы $t_k$ распределена по-другому,
    получаем противоречие с тем, что непонятно куда деть работу $t_k$.
\end{enumerate}

\subsection*{Задача 15.}
В задаче подразумевается, что имеется $k+1$ страница и $n$ запросов к ним.
Пусть изначально в кэше лежат элементы $1, \ldots, k$. Пусть при обращении
к первому элементу произошло вытеснение. Так как вы вытеснили элемент,
к которому обращение произойдет позже всего (и только на этом элементе
и случится новый промах) -- это означает,
что ко всем остальным элементам (которых $k-1$) мы как минимум раз обратимся.
Таким образом на каждый промах (кроме быть может последнего) приходится $k-1$
попадание.

\subsection*{Задача 16.}
Доказательство для стратегии FIFO аналогично доказательству для стратегии LRU.
Надо разбить последовательность запросов на фазы между ошибками оптимального
алгоритма. Несложно понять, что за время одной фазы, FIFO сделает не более $k$
промахов.



\end{document}
